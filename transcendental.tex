\documentclass{article}
\usepackage[utf8]{inputenc}
\usepackage{amsmath}
\usepackage{amsthm}
\usepackage{amssymb}
\usepackage{booktabs}
\usepackage{csquotes}
\usepackage[margin=1in]{geometry}
\usepackage{url}

\renewcommand{\pod}[1]{\allowbreak\mathchoice%
  {\if@display \mkern18mu\else \mkern8mu\fi (#1)}
  {\if@display \mkern18mu\else \mkern8mu\fi (#1)}
  {\mkern4mu (#1)}
  {\mkern4mu (#1)}
}


\title{About comprehensive quantity classes of which the value is nor algebraic nor reducible to algebraic irrational numbers}
\date{1851}
\author{Liouville, J.\thanks{%
    Translator of this article: John Demessemaeker, Bernardo Meurer}}

\begin{document}
\maketitle
\newpage

\paragraph{1.}
A long time ago, I gave a presentation about this subject at the Science
Academy. I also had two Notes inserted in volume XVIII of Comptes rendus
(pages 883 and 910; sessions of the 13th and 20th of May 1844). I believe these
two Notes need to be reproduced and completed here. The first Note is formulated
as the following:
\begin{displayquote}
  To give examples of continued fractions of which we can prove that their
  value is not at the root of any algebraic equation
    $$f(x)=ax^n + bx^{n-1} + \cdots + gx + h = 0$$
  $a, b, \ldots, g$ being integers, it is sufficient to recollect that
  $\frac{p_0}{q_0}$ and $\frac{p}{q}$, being two successive reductions of the
  continued fraction that expresses the development of an immeasurable $x$ of
  this equation, the incomplete quotient $\mu$, that comes after the reduction
  $\frac{p}{q}$, serves to form the next reduction, ends up (that results from
  a Lagrange formula, referencing the Memoires of Berlin, 1768) being, for
  very big $q$ values, constantly inferior to
    $$\pm\frac{df(p,q)}{qf(p,q)dp}$$
  essentially a positive expression where we assume
    $$f(p,q)=q^n f(\frac{p}{q}) = ap^n + bp^{n-1} q + \cdots + hq^n$$
  Given that the abstraction is made of signs, we have now, with more certainty,
    $$\mu < \frac{df(p,q)}{qdp}$$
  since $f(p, q)$ is an equal integer at least to the unity if we admit (what
  is allowed) that the equation
    $$f(x) = 0$$
  has been stripped of any measurable factor; $f(p,q) = 0$ would give, indeed,
    $$f(\frac{p}{q}) = 0$$
  Now represented by $f'(x)$ the differential of $f(x)$, the inequality above
  will become
    $$\mu < q^{n-2}f'(\frac{p}{q})$$
  However, $f'(\frac{p}{q})$ is a finite quantity going towards the limit
  $f'(x)$, like $\frac{p}{q}$ to the limit $x$. Designating with $A$ a certain
  fixed number,which is superior to this limit, we will be sure to have
    $$\mu < Aq^{n-2}$$

  Thereby, the incomplete quotients of a continued fraction representing the
  root $x$ of an algebraic equation of the degree $n$, in rational coefficients,
  are subject to never pass the product of a certain constant number by the
  force $(n-2)$\textsuperscript{nd} of the denominator of the previous
  reduction.

  It is sufficient to give the incomplete quotients $\mu$ a mode of formation
  that makes them grow beyond a predetermined term, to obtain continued
  fractions of which the value will not be able to satisfy any algebraic
  equation itself; it will happen, for example, that if, starting with any first
  incomplete quotient, we form every following $\mu$ with the help of the
  previous reduced $\frac{p}{q}$, according to the law $\mu = q^q$, or according
  to the law $\mu = q^m$, $m$  being the index of the $\mu$ row.

  Moreover, the previous method, which offered the first, is not the only nor
  is it the simplest that we can use. Let’s add that there are also analogue
  theorems for ordinary series. In particular, we cite the series
    $$\frac{1}{l} + \frac{1}{l^{1 \cdot 2}} + \frac{1}{l^{1\cdot 2\cdot 3}}
      + \cdots + \frac{1}{l^{1\cdot 2\cdot 3\cdots m}} + \cdots$$
  $l$ being a whole number.
\end{displayquote}

\paragraph{2.}
The second Note contains a new and more simple demonstration of the theorem to
which I was led by the Lagrange formula. The real power of our method, as we
will see, is independent of this formula.
\begin{displayquote}
  If $x, x_1, x_2, \ldots, x_{n-1}$ are the $n$ roots (the first real, the others real or
  imaginary) of the algebraic equation
    $$f(x) = ax^n + bx^{n-1} + \cdots + gx + h = 0$$
  which we can assume is irreducible, and where $a, b, \ldots, g, h$ are
  integers that are either positive, negative or zero, as we wish. Let’s
  designate by $\frac{p_0}{q_0}$, $\frac{p}{q}$ two consecutive reductions of
  the continued fraction in which $x$ develops; and by $z$ the complete quotient
  that comes after, so we have
    $$\frac{p}{q} - x = \pm\frac{1}{q(qz + q_0)}$$
  Finally, stating
    $$f(p,q) = q^n f(\frac{p}{q}) = ap^n + bp^{n-1}q + \cdots + hq^n$$

  By the decomposition of $f(\frac{p}{q})$ in factors, with the help of the
  roots $x, x_1, \ldots, x_{n-1}$, we find
    $$\frac{p}{q} - x = \pm\frac{1}{q(qz+q_0)}
      = \frac{f(p,q)}{q^n\cdot a\left(\frac{p}{q}
      - x\right)\cdots\left(\frac{p}{q}-x_{n-1}\right)}$$
  However, in order to converge the reduction $\frac{p}{q}$ towards $x$, the
  quantity
    $$a\left(\frac{p}{q} - x_1\right)\cdots\left(\frac{p}{q} - x_{n-1}\right)$$
  also converges towards a finite limit,
    $$a(x-x_1)\cdots(x-x_{n-1})$$
  there is thus a certain maximum $A$ below which the limit will always remain.
  On the other hand, $f(p,q)$ is a whole number, at least equal to the unity,
  abstraction made of the sign. We have thus
    $$\frac{1}{q(qz+q_0)} > \frac{1}{Aq^n}$$
  of which
    $$z < Aq^{n-2} - \frac{q_0}{q} < Aq^{n-2}$$
  inequality subsists, even more so, when we substitute the complete $z$
  quotient of the integer part that it contains, namely the incomplete quotient
  $\mu$. The theorem we had in mind is hereby proved simply, without having to
  fall back on the Lagrange formula that we used earlier. We can, incidentally,
  apply a similar method to diverse development genres of which the irrational
  quantities are susceptible, and obtain that way interesting results.
\end{displayquote}

\paragraph{3.}
Let’s add some developments to what preceded. Still regarding a real root $x$ of
the equation, irreducible and with whole coefficients,
  $$f(x)=ax^n + bx^{n-1} + \cdots + gx + h = 0$$
which, if $n>1$, will also have these other roots $x_1, \ldots, x_{n-1}$,
essentially irrational or imaginary and different to $x$. But let’s stop using,
to get closer and closer to $x$, reductions of continued fractions, and let’s
use any fraction $\frac{p}{q}$. In doing so, as here above,
  $$f(p,q) = ap^n + bp^{n-1} q + \cdots + hq^n$$
we will be sure, if $n>1$, that the absolute value of the integer $f(p,q)$ is
at least equal to the unity, and we can again use the equation
  $$\frac{p}{q} - x = \frac{f(p, q)}{q^n\cdot a\left(\frac{p}{q} - x_1\right)
  \cdots\left(\frac{p}{q}- x_{n-1}\right)}$$
this consequence, by designating by $A$ a certain fixed number, we must have
(abstraction made of the sign) for all fractions $\frac{p}{q}$ which we now use,
  $$\frac{p}{q} - x > \frac{1}{Aq^n}$$
But the case of $n=1$ has to be, in turn, examined closely. This case could not
be presented right away; because if we assume the continued fraction, in which
we developed $x$, made up of an infinite number of terms, we had irrational $x$
and $n > 1$. But here, still assuming infinite numbers of successive fractions
$\frac{p}{q}$ of which $x$ is the limit, we need to consider the case $n = 1$
as possible.

To deal with this case, whether
  $$f(x) = ax + b = 0$$
or
  $$\frac{p}{q} - x = \frac{ap + bq}{aq}$$
Should it be possible that the numerator $ap + bq$ were zero, we would not be
able to draw any conclusion. But if we have ensured it by any way that we never
have
  $$ap + bq = 0$$
namely
  $$x = \frac{p}{q}$$
we could then argue that we have
  $$\frac{p}{q} - x > \frac{1}{aq}$$
or even
  $$\frac{p}{q} - x > \frac{1}{Aq}$$
writing $A$ instead of $a$. The general formula
  $$\frac{p}{q} - x > \frac{1}{Aq^n}$$
will thus subsist even in the case of $n=1$.

That being said, granting that the quantity $x$ as such, forming infinite
numbers of fractions $\frac{p}{q}$ going towards $x$, but of which none are
exactly equal to $x$, we end up recognizing that the inequality
  $$\frac{p}{q} - x > \frac{1}{Aq^n}$$
does not always occur. It has to be concluded that $x$ cannot be the root of
an equation of the degree $n$. Let’s also add that $x$ will not be root of any
equation of inferior degree $i$; because with
  $$\frac{p}{q} - x > \frac{1}{Aq^n}$$
we will have, a fortiori,
  $$\frac{p}{q} - x > \frac{1}{Aq^i}$$
for every exponent $i < n$. Therefore, having determined that the inequality
  $$\frac{p}{q} - x > \frac{1}{Aq}$$
is in default, we will conclude that the value x is not rational. If the higher
inequality
  $$\frac{p}{q} - x > \frac{1}{Aq^2}$$
is inadmissible, we will conclude that $x$ is not rational, nor even the root
of a second-degree equation; and so on. Finally, should it happen that the
inequality
  $$\frac{p}{q} - x > \frac{1}{Aq^n}$$
generally takes place in default, any finite number we select for $n$, we will
be able to claim that $x$ is not even an algebraic irrational.

\paragraph{4.}
If, for example,
  $$x = \frac{1}{l} + \frac{1}{l^{1 \cdot 2}} + \frac{1}{l^{1\cdot 2\cdot 3}}
    + \cdots + \frac{1}{l^{1\cdot 2\cdot 3\cdots m}} + \cdots$$
$l$ being a whole number. Contributing to $\frac{p}{q}$ the rough value, but
essentially too small of $x$, that give the $m$’s the first terms of the series,
we will have
  $$q = l^{1\cdot 2\cdots m}$$
and
  $$x - \frac{p}{q} = \frac{1}{l^{1\cdot 2\cdots m(m + 1)}} + \cdots
  < \frac{2}{q^{m+1}}$$
If the exponent $m$ grows indefinitely, this last quantity decreases faster
than any fraction having a constant numerator and a proportionate denominator of
a given power of $q$, so that the inequality
  $$\frac{p}{q} - x > \frac{1}{Aq^n}$$
always ends up being in default. From this, I conclude that $x$ is not rational,
nor even expressible by algebraic irrationals.

We will easily get to a similar conclusion for the much more general series
  $$x = \frac{k_1}{l} + \frac{k_2}{l^{1\cdot 2}} + \frac{k_3}{l^{1\cdot 2\cdot 3}}
    + \cdots + \frac{k_m}{l^{1\cdot 2\cdots m}} + \cdots$$
$k_1, k_2, k_3, \ldots, k_m, \ldots$ designating whole numbers, positive or
negative, of which the absolute value does not exceed a certain maximum $k$.
So if we take $l = 10$, and $k_1, k_2, \ldots$, freely, from $0$ until $9$,
we will form undefined decimal fractions of which the value will never be able
to expressed algebraically. I believe I remember that there is such a theorem,
in a letter from Goldbach to Euler; but I do not know whether its proof has
been provided.

Now let's say
  $$x = \frac{1}{l} + \frac{1}{l^4} + \frac{1}{l^9} + \cdots + \frac{1}{l^{m^2}}
    + \cdots$$
Still assuming for the rough value of $x$ the sum of the first $m$ terms of the
series, we will have
  $$q = l^{m^2}$$
and
  $$x - \frac{p}{q} = \frac{1}{l{(m + 1)}^2} + \cdots
    < \frac{2}{l^{2m+1}\cdot q}$$
This last quantity, the denominator containing the product of $q$ by $l^{2m+1}$
that grows with $m$, will end up decreasing faster than $\frac{1}{Aq}$. But the
only thing we can conclude from that is that $x$ is not rational.

As a new example, if
  $$x = \frac{1}{l} + \frac{1}{l_1} + \frac{1}{l_2} + \cdots
    + \frac{1}{l_{m-1}} + \cdots$$
$l$ being an integer, and every term has as a denominator the
$(n + 1)$\textsuperscript{th} power of the previous denominator, so that
  $$l_m = l_{m-1}^{n+1}$$
It is obvious that $l_{m-1}$ will be a power of $l$, consequently, we will have
  $$q = l_{m-1}$$
and
  $$x - \frac{p}{q}=\frac{1}{l_m}+\cdots < \frac{2}{q^{n+1}}$$
$x$ will thus not be the root of any algebraic equation of a degree equal or
inferior to $n$.

We would also have other examples, considering the series
  $$  x = \frac{1}{l} + \frac{1}{u_1} + \frac{1}{u_1 l_2} + \cdots
    + \frac{1}{u_1 l_2 \cdots l_{m-1}} + \cdots$$
where $l, l_1, l_2, \ldots, l_{m-1}, \ldots$ designate increasingly bigger whole
numbers. This circumstance where $l_m$ grows beyond any limit means that $x$
could no longer be rational. And if $l_m$ grows fast enough with the $m$ index,
we will be sure that $x$ is not even an algebraic irrational.

\paragraph{5.}
Finally, let us observe that, if we assume $a, b, \ldots, g, h$ as imaginary
and complex integers, then move towards a imaginary root $x$ of the equation
  $$ax^n + bx^{n-1} + \cdots + gx + h = 0$$
with the help of fractions $\frac{p}{q}$ of which the two terms would also be
complex integers, we would again find the equation
  $$\frac{p}{q} - x = \frac{f(p,q)}{q^n\cdot a\left(\frac{p}{q}
    - x_1\right)\cdots\left(\frac{p}{q}-x_{n-1}\right)}$$
and substituting the imaginary’s modules, of which we would easily deduce
  $$\mod\left(\frac{p}{q}- x\right) > \frac{1}{A\pmod{q}^n}$$
which allows to extend the results of the imaginaries that we just developed for
real quantities. This way, we recognize, for example, that, whichever be the
complex integer $l$, the sum of the series
$$\frac{1}{l} + \frac{1}{l^{1 \cdot 2}} + \frac{1}{l^{1\cdot 2\cdot 3}}
  + \cdots + \frac{1}{l^{1\cdot 2\cdot 3\cdots m}} + \cdots$$
is never algebraically expressible.
\end{document}
